\documentclass{beamer}


\usetheme{Berlin}
\usecolortheme{default}


\title{Introduction to LaTeX and Beamer}
\author{Mohammed Pelkhier}
\date{\today}

\begin{document}

\begin{frame}
  \titlepage
\end{frame}


\begin{frame}
  \frametitle{What is LaTeX?}
  LaTeX is a typesetting system used for the production of documents with high-quality typesetting. It is widely used in academia for scientific papers, reports, and presentations.

  \vspace{0.5cm}
  
  LaTeX provides a powerful and consistent framework for document creation, ensuring professional-looking results.
\end{frame}


\begin{frame}
  \frametitle{Why use LaTeX?}
  \begin{itemize}
    \item LaTeX produces professional-looking documents with consistent formatting.
    \item It separates content from presentation, allowing you to focus on the content.
    \item LaTeX is highly customizable and supports advanced typesetting features.
    \item It has extensive support for mathematical equations and symbols.
  \end{itemize}
\end{frame}


\begin{frame}
  \frametitle{What is Beamer?}
  Beamer is a LaTeX package specifically designed for creating presentations. It provides various features like slide transitions, overlays, and animation effects.

  \vspace{0.5cm}
  
  With Beamer, you can create professional-looking presentations in LaTeX, maintaining consistency with your other documents.
\end{frame}


\begin{frame}
  \frametitle{Creating Slides with Beamer}
  \begin{itemize}
    \item Start by defining the document class as \texttt{beamer}.
    \item Choose a theme and color scheme for the presentation.
    \item Use the \texttt{frame} environment to create individual slides.
    \item Add content and structure to each slide using commands like \texttt{frametitle} and \texttt{itemize}.
  \end{itemize}
\end{frame}


\begin{frame}
  \frametitle{Slide Transitions and Effects}
  Beamer allows you to add slide transitions and effects to your presentation. For example:
  \begin{itemize}
    \item Use \texttt{pause} to reveal content step-by-step.
    \item Add \texttt{<+->} after an itemize/enumerate environment for automatic slide transitions.
    \item Use commands like \texttt{only}, \texttt{uncover}, and \texttt{alert} for highlighting specific content.
  \end{itemize}

  \vspace{0.5cm}
  
  Let's see some examples on the next slide.
\end{frame}


\begin{frame}
  \frametitle{Examples of Slide Transitions and Effects}
  \begin{itemize}
    \item This line will be visible from the beginning.
    \item This line will appear after the first item.
    \pause
    \item This line will appear after the second item.
    \item This line will appear after the third item.
  \end{itemize}
  
  \vspace{0.5cm}
  
  \onslide<4->{This line will appear from the fourth slide onwards.}
\end{frame}


\begin{frame}
  \frametitle{Mathematical Equations}
  LaTeX is renowned for its support for typesetting mathematical equations and symbols. You can use packages like \texttt{amsmath} and \texttt{amssymb} for enhanced mathematical typesetting.

  \vspace{0.5cm}

  Some equation examples:
  \begin{align*}
    E &= mc^2 \\
    \sum_{n=1}^{\infty} \frac{1}{n^2} &= \frac{\pi^2}{6} \\
    \int_{0}^{1} x^2 \, dx &= \frac{1}{3}
  \end{align*}
\end{frame}


\begin{frame}
  \frametitle{Conclusion}
  LaTeX, together with the Beamer package, offers a powerful and flexible solution for creating professional presentations. Its ability to separate content from presentation allows you to focus on the substance of your talk while producing visually appealing slides.
\end{frame}

\end{document}
